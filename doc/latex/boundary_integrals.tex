% Created 2015-09-16 Wed 21:53
\documentclass[11pt]{article}
\usepackage[utf8]{inputenc}
\usepackage[T1]{fontenc}
\usepackage{fixltx2e}
\usepackage{graphicx}
\usepackage{longtable}
\usepackage{float}
\usepackage{wrapfig}
\usepackage{rotating}
\usepackage[normalem]{ulem}
\usepackage{amsmath}
\usepackage{textcomp}
\usepackage{marvosym}
\usepackage{wasysym}
\usepackage{amssymb}
\usepackage{hyperref}
\tolerance=1000
\usepackage[margin=1in]{geometry}
\parindent=0in \parskip=0.5\baselineskip
\author{Constantine Khroulev}
\date{\today}
\title{Evaluating line and surface integrals to implement boundary conditions}

\begin{document}

\maketitle
\tableofcontents

\newcommand{\dx}{\Delta x}
\newcommand{\dy}{\Delta y}
\newcommand{\Real}{\mathbb{R}}
\newcommand{\diff}[2]{\frac{\partial #1}{\partial #2}}
\newcommand{\Diff}[2]{\frac{d #1}{d #2}}
\newcommand{\rs}{\diff{\Fk{\R}}{s}}
\newcommand{\rt}{\diff{\Fk{\R}}{t}}
\newcommand{\R}{\mathbf{r}}
\newcommand{\N}{\mathbf{n}}

\section{Introduction}
\label{sec-1}

To implement Neumann boundary conditions in a FEM solver we need to be
able to evaluate integrals over element boundaries. In the
two-dimensional case this requires computing \emph{line integrals}; in
the three-dimensional case --- \emph{surface integrals}. In these
notes I describe how to use element maps to approximate these
integrals using quadratures.

Karniadakis and Sherwin mention this in section 4.1.4 (page 160) of
\cite{KarniadakisSherwin}, but use an unhelpful name \emph{surface
  Jacobian} and don't go into details.

\subsection{Notation}
\label{sec:notation}

In these notes quantities with the subscript $\cdot_{*}$ relate to the reference
element, $\cdot_{k}$ to the $k$-th physical element. Superscripts
$\cdot^{j}$ relate to the $j$-th \emph{boundary} (side or face) of an element.

So, $E_{*}$ is the reference element itself, $E_{*}^{1}$ is its first
side, $x_{k}$ is the function mapping from the coordinates in the
system used by the reference element to the $x$-coordinate in the
$k$-th element, while $x_{k}^{j}$ maps the $j$-th side of the reference
element to the corresponding side of the physical element. A subscript
$\cdot_{k,j}$ refers to the $j$-th \emph{node} of the element $k$.

\section{Preliminaries}
\label{sec-2}

\subsection{Line integrals}
\label{sec-2-1}

For some scalar function $F : U \subset \Real^{n} \to \Real$
the integral along a piecewise smooth curve $C \subset U$ is defined as
\begin{equation}
\label{eq:1}
\int_{C} F\, ds = \int_{a}^{b} F(\R(t))\, \left| \frac{d}{dt}\R(t) \right|\, dt,
\end{equation}
where $\R: [a, b] \to C$ is a bijective parameterization of the curve
$C$ \cite{Nikolsky1977}.

\subsection{Surface integrals}
\label{sec-2-2}

Let $\Omega \subset \Real^{2}$ be a measurable subset of the
plane and define $\R(s,t) = (x(s,t), y(s,t), z(s,t))$, where $x(s,
t)$, $y(s, t)$, and $z(s, t)$ are continuously differentiable
functions defined on $\Omega$. Then $S = \R(\Omega) \subset
\Real^{3}$ is a smooth surface parameterized by $\R$.

Vectors $\R_{s}$ and $\R_{t}$ are tangential vectors to the surface $S$
given by the parameterization $\R(s,t)$.

If $\R_{s}$ and $\R_{t}$ are not parallel, then $\N = \R_{s} \times \R_{t}$ is
a normal vector, $| \N | > 0$, and we can define
the integral
\begin{equation}
\label{eq:2}
\int_{S} F\, dS = \int_{\Omega} F\, |\N|\, ds dt.
\end{equation}

Here
\begin{equation}
\label{eq:3}
\R_{s} = \left(\diff{x(s,t)}{s}, \diff{y(s,t)}{s}, \diff{z(s,t)}{s}\right) \quad \text{and} \quad
\R_{t} = \left(\diff{x(s,t)}{t}, \diff{y(s,t)}{t}, \diff{z(s,t)}{t}\right).
\end{equation}
See \cite{Nikolsky1977} or any other Calculus textbook for details.


\section{Boundaries of quadrilateral elements}
\label{sec-3}
\newcommand{\F}[2]{#1_{#2}^{1}}
\newcommand{\Fk}[1]{\F{#1}{k}}
\newcommand{\Fr}[1]{\F{#1}{*}}
\newcommand{\T}{\Fr{\R}(t)}
\newcommand{\Tq}{\Fr{\R}(t_{q})}
\newcommand{\Tkq}{\Fk{\R}(t_{q})}

Consider the reference element $E_{*} = [-1, 1] \times [-1,1]$ with nodal coordinates
$(\xi_{*,i}, \eta_{*,i}), i=1,\dots,4$.
We number faces of the reference element as follows.
\begin{equation}
\label{eq:4}
\begin{aligned}
E_{*}^{1} &=  E_{*} \cap \left \{ (\xi, \eta)\, |\, \xi = -1 \right \},\\
E_{*}^{2} &=  E_{*} \cap \left \{ (\xi, \eta)\, |\, \xi = 1  \right \},\\
E_{*}^{3} &=  E_{*} \cap \left \{ (\xi, \eta)\, |\, \eta = -1 \right \},\\
E_{*}^{4} &=  E_{*} \cap \left \{ (\xi, \eta)\, |\, \eta = 1  \right \}.
\end{aligned}
\end{equation}

The first side $\Fr{E}$ is parameterized by
\begin{equation}
  \label{eq:5}
  \begin{aligned}
      \Fr{\R}(t) &= \left(\Fr{\xi}(t), \Fr{\eta}(t) \right),\\
      \Fr{\xi}(t) &= -1,\\
      \Fr{\eta}(t) &= t.
  \end{aligned}
\end{equation}
(Definitions of $\Fr{\R}$ for $j = 2, 3, 4$ are very similar and
are omitted.)

Consider a physical element $E_{k}$ with nodal coordinates $(x_{k,i}, y_{k,i}),\, i = 1,\dots,4$.
A 2D FEM construction includes the map from the reference element
$E_{*} = [-1, 1] \times [-1,1]$ to $E_{k}$:
\begin{equation}
\label{eq:6}
\R_{k}(\xi,\eta) = \left( x_{k}(\xi,\eta), y_{k}(\xi,\eta) \right).
\end{equation}

A side of the reference element is mapped to the corresponding side of
the physical element. So, we can use the parameterization of
$\Fr{E}$ (\ref{eq:5}) and the map $\R_{k}$ (\ref{eq:6}) to create a parameterization of
$\Fk{E}$:
\begin{equation}
  \label{eq:7}
  \Fk{\R}(t) = \R_{k}(\Fr{\xi}(t), \Fr{\eta}(t)),\\
\end{equation}
Note that (\ref{eq:7}) uses a composition of functions, and we have to
use the \emph{full derivative} to compute $\frac{d}{dt}\Fk{\R}$:
\begin{equation}
  \label{eq:8}
  \Diff{}{t}\Fk{\R} = \diff{\Fk{\R}}{\Fr{\xi}} \Diff{\Fr{\xi}}{t} + \diff{\Fk{\R}}{\Fr{\eta}} \Diff{\Fr{\eta}}{t},
\end{equation}
but it reduces to a partial derivative with respect to the variable
replaced by $t$ because the other variable is fixed:
\begin{equation}
  \label{eq:9}
  \begin{aligned}
  \Diff{}{t}\Fk{\R} &= \diff{\Fk{\R}}{\Fr{\xi}}\cdot 0 + \diff{\Fk{\R}}{\Fr{\eta}} \cdot 1,\\
    &=\diff{\Fk{\R}}{\Fr{\eta}}.
  \end{aligned}
\end{equation}
\subsection{Example: evaluating integrals over sides of $Q_{1}$ quad elements}
\label{sec-3-1}

In this example I use the first (``left'') side, but computations for
all sides are similar.

Let $(\xi_{*,i}, \eta_{*,i}), i = 1,\dots,4$ be coordinates of the nodes
of the reference element $E_{*}$ and $(x_{k,i}, y_{k,i})$ be coordinates
of the nodes of a physical element $E_{k}$.

Let $F(x,y)$ be a function defined on the FEM mesh and on $\Fk{E}$ in
particular. Assume that we know its nodal values
\begin{equation}
\label{eq:10}
F_{k,i} = F(x_{k,i}, y_{k,i}).
\end{equation}

Our goal is to approximate the integral
\begin{equation}
\label{eq:11}
\Fk{I} = \int_{\Fk{E}} F(x,y)\, ds
\end{equation}

In this case the map $\R_{k}$ is defined by 
\begin{equation}
\label{eq:12}
\begin{aligned}
\R_{k}(\xi,\eta) &= \left( x_{k}(\xi,\eta), y_{k}(\xi,\eta) \right),\\
x_{k}(\xi,\eta) &= \sum_{i=1}^{4} x_{k,i}\, \chi_{i}(\xi, \eta),\\
y_{k}(\xi,\eta) &= \sum_{i=1}^{4} y_{k,i}\, \chi_{i}(\xi, \eta),\\
\end{aligned}
\end{equation}
where
\begin{equation}
\label{eq:13}
\chi_{i}(\xi,\eta) = \frac14 (1 + \xi\,\xi_{*,i})(1 + \eta\,\eta_{*,i})
\end{equation}
are $Q_{1}$ reference element basis functions.

We combine (\ref{eq:5}), (\ref{eq:6}), and (\ref{eq:7}) to define $\Fk{\R}(t)$, the parameterization of the first side:
\begin{equation}
  \label{eq:14}
  \Fk{\R}(t) = \left( x_{k}(\T),\, y_{k}(\T) \right).
\end{equation}

Now we can rewrite $\Fk{I}$
\begin{equation}
\label{eq:15}
\Fk{I} = \int_{-1}^{1} F(\Fk{\R}(t))\, \left|\frac{d}{dt}\Fk{\R}(t) \right|\, dt.
\end{equation}

Using (\ref{eq:12}) we can re-write $\frac{d}{dt} \Fk{\R}(t)$:
\begin{equation}
\label{eq:16}
\frac{d}{dt} \Fk{\R}(t) = \left( \sum_{i=1}^{4} x_{k,i}\, \frac{d}{dt}\chi_{i}(\T),\,
                                 \sum_{i=1}^{4} y_{k,i}\, \frac{d}{dt}\chi_{i}(\T)  \right).
\end{equation}

Differentiating $\chi_{i}(\T)$ gives
\begin{equation}
  \label{eq:17}
  \begin{aligned}
      \frac{d}{dt}\chi_{i}(\T) &= \frac{d}{dt}\chi_{i}(\Fr{\xi}(t), \Fr{\eta}(t))\\
      &=\diff{\chi_{i}(\Fr{\xi}(t), \Fr{\eta}(t))}{\Fr{\xi}}\,\diff{\Fr{\xi}(t)}{t} +
      \diff{\chi_{i}(\Fr{\xi}(t), \Fr{\eta}(t))}{\Fr{\eta}}\,\diff{\Fr{\eta}(t)}{t}\\
      &= \diff{\chi_{i}(\Fr{\xi}(t), \Fr{\eta}(t))}{\Fr{\eta}}\\
      &=\diff{\chi_{i}(\xi, \eta)}{\eta}(-1, t).
  \end{aligned}
\end{equation}

This simplifies (\ref{eq:16}) to
\begin{equation}
  \label{eq:18}
\frac{d}{dt} \Fk{\R}(t) = \left( \sum_{i=1}^{4} x_{k,i}\,\diff{\chi_{i}(\xi, \eta)}{\eta}(-1, t),\,
                                 \sum_{i=1}^{4} y_{k,i}\,\diff{\chi_{i}(\xi, \eta)}{\eta}(-1, t) \right).
\end{equation}

The integral in \eqref{eq:15} is a \emph{regular} definite integral over
the interval $[-1, 1]$ and can be approximated using a quadrature.

\begin{equation}
\label{eq:19}
\Fk{I} \approx \sum_{q=1}^{N_{q}} w_{q}\, F(\Tkq), y(\Tkq))\, \left|\frac{d}{dt}\Fk{\R}(t_{q}) \right|.
\end{equation}

Now we just need to approximate $F(\Fk{\R}(t_{q}))$ by
a bilinear function built using $Q_{1}$ basis functions \eqref{eq:13}:
\begin{equation}
\label{eq:20}
F(\Fk{\R}(t)) \approx \sum_{j=1}^{4} F_{k,j}\, \chi_{j}(\T)
\end{equation}
so that
\begin{equation}
\label{eq:21}
\Fk{I} \approx \sum_{q=1}^{N_{q}} w_{q}\,
\left( \sum_{j=1}^{4} F_{k,j}\, \chi_{j}(\Tq) \right) \, \left| \frac{d}{dt}\Fk{\R}(t_{q}) \right|.
\end{equation}

In the equal spacing case ($x_{4} = x_{1}$, $x_{2} = x_{3} = x_{1} +
\dx$, $y_{2} = y_{1}$, $y_{3} = y_{4} = y_{1} + \dy$) the factor
$\left| \frac{d}{dt}\R_{k}^{j}\right|, j=1,\dots,4$ can be simplified to
\begin{align*}
\left| \frac{d}{dt}\R_{k}^{1}\right|  & = \frac{\dy}{2}, \\
\left| \frac{d}{dt}\R_{k}^{2}\right|  & = \frac{\dy}{2}, \\
\left| \frac{d}{dt}\R_{k}^{3}\right|  & = \frac{\dx}{2}, \\
\left| \frac{d}{dt}\R_{k}^{4}\right|  & = \frac{\dx}{2}. \\
\end{align*}

\section{Boundaries of hexahedral elements}
\label{sec-4}
\renewcommand{\Tq}{\Fr{\R}(s_{q}, t_{q})}
\renewcommand{\Tkq}{\Fk{\R}(s_{q}, t_{q})}

A 3D FEM construction includes the map from the reference element
$E_{*} = [-1, 1] \times [-1,1] \times [-1,1]$ to a physical element:

\begin{equation}
\label{eq:22}
\R_{k}(\xi,\eta,\zeta) = \left( x_{k}(\xi,\eta,\zeta), y_{k}(\xi,\eta,\zeta), z_{k}((\xi,\eta,\zeta) \right)
\end{equation}

Note that $\R_{k}$ maps a face of the reference element to the
corresponding face of a physical element. We number the faces of the
reference element as follows.

\begin{equation}
\label{eq:23}
\begin{aligned}
E_{*}^{1} &=  E_{*} \cap \left \{ (\xi, \eta, \zeta)\, |\, \xi = -1 \right \},\\
E_{*}^{2} &=  E_{*} \cap \left \{ (\xi, \eta, \zeta)\, |\, \xi = 1  \right \},\\
E_{*}^{3} &=  E_{*} \cap \left \{ (\xi, \eta, \zeta)\, |\, \eta = -1 \right \},\\
E_{*}^{4} &=  E_{*} \cap \left \{ (\xi, \eta, \zeta)\, |\, \eta = 1  \right \},\\
E_{*}^{5} &=  E_{*} \cap \left \{ (\xi, \eta, \zeta)\, |\, \zeta = -1 \right \},\\
E_{*}^{6} &=  E_{*} \cap \left \{ (\xi, \eta, \zeta)\, |\, \zeta = 1  \right \}.
\end{aligned}
\end{equation}

\renewcommand{\F}[2]{#1_{#2}^{5}}
We can use (\ref{eq:23}) to define parameterizations of faces of the
reference element. For example, the fifth face is parameterized by
\begin{equation}
  \label{eq:24}
  \begin{aligned}
    \Fr{\R}(s,t) &= \left( \Fr{\xi}(s,t), \Fr{\eta}(s,t), \Fr{\zeta}(s,t) \right),\\
    \Fr{\xi}(s,t) &= s,\\
    \Fr{\eta}(s,t) &= t,\\
    \Fr{\zeta}(s,t) &= -1.
  \end{aligned}
\end{equation}

The faces of a physical element are images of reference element faces
under $\R_{k}$: $E_{k}^{i} = \R_{k}(E_{*}^{i})$. The face $\Fk{E}$ is
parameterized by
\begin{equation}
\label{eq:25}
\begin{aligned}
  \Fk{\R}(s, t) &= \R_{k}(\Fr{\R}(s,t)),\\
  (s,t) &\in D,\\
  D &= [-1, 1] \times [-1, 1].
\end{aligned}
\end{equation}
and so on.

\subsection{Example: evaluating an integral over a face of a $Q_{1}$ element.}
\label{sec-4-1}

\newcommand{\face}{\Fk{E}}

In this example I use the fifth, or ``bottom'' face, but all
computations are very similar in the five other cases.

Let $(\xi_{*,i}, \eta_{*,i}, \zeta_{*,i})$, $i = 1,\dots,8$ be coordinates
of the nodes of the reference element $E_{*}$ and $(x_{k,i}, y_{k,i},
z_{k,i})$ the coordinates of the nodes of a physical element $E_{k}$.

Let $F(x,y,z)$ be a function defined on the FEM mesh and on $\face$
in particular. Assume that we know its nodal values
\begin{equation*}
F_{k,i} = F(x_{k,i}, y_{k,i}, z_{k,i}).
\end{equation*}

Our goal is to approximate the integral
\begin{equation}
\label{eq:26}
\Fk{I} = \int_{\face} F(x,y,z)\, dS.
\end{equation}

In this case the map $\R_{k} = \left( x_{k}(\xi,\eta,\zeta), y_{k}(\xi,\eta,\zeta), z_{k}(\xi,\eta,\zeta) \right)$ (\ref{eq:22}) is defined by
\begin{equation}
\label{eq:27}
\begin{aligned}
x_{k}(\xi,\eta,\zeta) &= \sum_{i=1}^{8} x_{k,i}\, \chi_{i}(\xi, \eta, \zeta),\\
y_{k}(\xi,\eta,\zeta) &= \sum_{i=1}^{8} y_{k,i}\, \chi_{i}(\xi, \eta, \zeta),\\
z_{k}(\xi,\eta,\zeta) &= \sum_{i=1}^{8} z_{k,i}\, \chi_{i}(\xi, \eta, \zeta),
\end{aligned}
\end{equation}
where
\begin{equation}
\label{eq:28}
\chi_{i}(\xi,\eta,\zeta) = \frac18 (1 + \xi\xi_{i})(1 + \eta\eta_{i})(1 + \zeta\zeta_{i})
\end{equation}
are $Q_{1}$ reference element basis functions.

We can now rewrite $\Fk{I}$ using \eqref{eq:22}, (\ref{eq:25}) \eqref{eq:27}, and the definition of the surface integral:
\begin{equation}
\label{eq:29}
\Fk{I} = \int_{D} F(\Fk{\R}(s,t))\, | \N(s,t) |\,ds\,dt,
\end{equation}
where $\N(s,t) = \rs(s,t) \times \rt(s,t)$ and
\begin{equation}
\label{eq:30}
\begin{aligned}
\rs(s,t) &= \left(\diff{x_{k}(\Fr{\R}(s,t))}{s}, \diff{y_{k}(\Fr{\R}(s,t))}{s}, \diff{z_{k}(\Fr{\R}(s,t))}{s}\right),\\
\rt(s,t) &= \left(\diff{x_{k}(\Fr{\R}(s,t))}{t}, \diff{y_{k}(\Fr{\R}(s,t))}{t}, \diff{z_{k}(\Fr{\R}(s,t))}{t}\right).
\end{aligned}
\end{equation}

Consider $\diff{}{s}x_{k}(\Fr{\R}(s,t))$ on $D$:
\newcommand{\xFr}{x_k(\Fr{\xi}, \Fr{\eta}, \Fr{\zeta})}
\begin{equation}
  \label{eq:31}
  \begin{aligned}
    \diff{}{s}x_{k}(\Fr{\R}(s,t)) &= \diff{}{s}\xFr\\
    &=\diff{\xFr}{\Fr{\xi}}\,\diff{\Fr{\xi}}{s}
    +\diff{\xFr}{\Fr{\eta}}\,\diff{\Fr{\eta}}{s}
    +\diff{\xFr}{\Fr{\zeta}}\,\diff{\Fr{\zeta}}{s}\\
    &=\diff{x_{k}}{\xi}(\Fr{\R}(s,t)) \times 1 + 0 + 0
  \end{aligned}
\end{equation}

The integral $\Fk{I}$ written in the form \eqref{eq:29} can be approximated
using a quadrature on $D$, i.e.
\begin{equation}
\label{eq:32}
\Fk{I} \approx \sum_{q=1}^{N_{q}} w_{q}\cdot F(\Tkq) | \N(s_{q},t_{q}) |,
\end{equation}
where $w_{q}$ are quadrature weights and $s_{q}, t_{q}$ are
coordinates of quadrature points.

To compute this quadrature we need $F(\Tkq)$, but we
assume that only nodal values of $F$ are available.

We use \eqref{eq:28} as a basis to write a trilinear approximation of $F$ on $\face$:
\begin{equation}
\label{eq:33}
F(s,t) \approx \sum_{k=1}^{8} F_{k}\,\chi_{k}(\Fr{\R}(s,t)),
\end{equation}
then
\begin{equation}
\label{eq:34}
\Fk{I} \approx \sum_{q=1}^{N_{q}} w_{q}\left( \sum_{k=1}^{8} F_{k}\,\chi_{k}(\Fk{\R}(s_q, t_q)) \right) | \N(s_{q},t_{q}) |,
\end{equation}

Finally, to evaluate $|\N(s_{q}, t_{q})| = \left| \rs(s_{q}, t_{q}) \times
\rt(s_{q}, t_{q}) \right|$ we use \eqref{eq:3} and \eqref{eq:27}.

From (\ref{eq:27}) and (\ref{eq:31}) we have
\begin{equation}
  \label{eq:35}
  \diff{}{s}x_{k}(\Fr{\R}(s,t)) = \sum_{j=1}^{8}x_{k,j}\diff{\chi_{j}}{\xi}(\Fr{\R}(s,t))
\end{equation}
and similarly for $y_{k}$, $z_{k}$, and $t$-derivatives.

\subsubsection{Summary}
\label{sec:summary}

Overall, here are the steps needed to approximate \eqref{eq:26}.

\begin{enumerate}
\item Pre-compute $\chi_{j}(\Tq)$ for $\face$ and similarly
for the remaining five faces.
\item Pre-compute $\diff{\chi_{i}(\Fr{\R})}{s}$ and
$\diff{\chi_{i}(\Fr{\R})}{t}$ at $(s_{q}, t_{q})$ for $\face$ and similarly for
other faces.
\end{enumerate}

Then for each quadrature point we need to

\begin{enumerate}
\item use pre-computed values to approximate $\rs(s_{q},t_{q})$ and
$\rt(s_{q},t_{q})$ (see \eqref{eq:35}),
\item compute $|\N(s_{q}, t_{q})| = \left| \rs(s_{q}, t_{q}) \times \rt(s_{q}, t_{q}) \right|$,
\item approximate $F(s_{q}, t_{q})$ using \eqref{eq:33},
\item multiply to evaluate a term of \eqref{eq:34}.
\end{enumerate}

\subsubsection{Equal horizontal spacing}
\label{sec-4-1-1}

The procedure described above can be simplified if we know that the
FEM mesh uses regular spacing in the $x$ and $y$ directions, i.e.
$x$ and $y$ coordinates of the nodes of a physical element are
\begin{equation}
  \label{eq:36}
  \begin{array}{rcllllllll}
    x_{i} &=& \{ x_{1}, &x_{1} + \dx, &x_{1} + \dx, &x_{1}, &x_{1}, &x_{1} + \dx, &x_{1} + \dx, &x_{1} \},\\
    y_{i} &=& \{ y_{1}, &y_{1}, &y_{1} + \dy, &y_{1} + \dy, &y_{1}, &y_{1}, &y_{1} + \dy, &y_{1} + \dy \};\\
  \end{array}
\end{equation}
$z_{i}$ are may all be different.

In this case \eqref{eq:30} simplifies to
\begin{align}
  \label{eq:37}
  \rs(s,t) &= \left(
             \frac{\dx}{2},\,
             0,\,
             \sum_{i=1}^{8} z_{i} \diff{\chi_{i}(s, t, -1)}{s} \right),\\
  \rt(s,t) &= \left(
             0,\,
             \frac{\dy}{2},\,
             \sum_{i=1}^{8} z_{i} \diff{\chi_{i}(s, t, -1)}{t} \right).
\end{align}

With these simplifications $\N(s,t)$ becomes

\begin{equation}
  \label{eq:38}
  \N(s,t) = \left(
    -\frac{\dy}{2}\diff{z(s,t,-1)}{s},
    -\frac{\dx}{2}\diff{z(s,t,-1)}{t},
    \frac{1}{4} \dx \dy
  \right)
\end{equation}

Note that if all $z_{i}$ are equal, $|\N| = \frac{1}{4} \dx \dy$ is
equal to the determinant of the Jacobian of the map from the reference
element to a physical element in the 2D $Q_{1}$ FEM setup with equal
spacing in $x$ and $y$ directions --- as expected.

The only thing that changes depending on the face we're integrating
over is the expression for $|\N|$. Below are formulas for all six
faces (assuming equal horizontal spacing and using the numbering
defined by \eqref{eq:23}).

\newcommand{\ddt}{\frac{d}{d\,t}}
\newcommand{\dds}{\frac{d}{d\,s}}
\begin{align*}
  \N_{1} &= \Big(\frac{1}{2}\dy\ddt\, z(-1, s, t), 0, 0  \Big), \\
  \N_{2} &= \Big(\frac{1}{2}\dy\ddt\, z(1, s, t), 0, 0  \Big), \\
  \N_{3} &= \Big(0, -\frac{1}{2}\dx\, \ddt\, z(s, -1, t), 0  \Big), \\
  \N_{4} &= \Big(0, -\frac{1}{2}\dx\, \ddt\, z(s, 1, t), 0  \Big), \\
  \N_{5} &= \Big(-\frac{1}{2}\dy\, \dds\, z(s, t, -1), -\frac{1}{2}\dx\, \ddt\, z(s, t, -1), \frac{1}{4}\dx\, \dy  \Big), \\
  \N_{6} &= \Big(-\frac{1}{2}\dy\, \dds\, z(s, t, 1), -\frac{1}{2}\dx\, \ddt\, z(s, t, 1), \frac{1}{4}\dx\, \dy  \Big) . \\
\end{align*}

\begin{align*}
  \left| \N_{1}\right|  & = \frac{1}{2}\dy\left| \ddt\,z(-1 , s , t)\right|, \\
  \left| \N_{2}\right|  & = \frac{1}{2}\dy\left| \ddt\,z(1 , s , t)\right|, \\
  \left| \N_{3}\right|  & = \frac{1}{2}\dx\left| \ddt\,z(s , -1 , t)\right|,\\
  \left| \N_{4}\right|  & = \frac{1}{2}\dx\left| \ddt\,z(s , 1 , t)\right|, \\
  \left| \N_{5}\right|  & = \sqrt{\frac{1}{4}\dx^2\,\left(\ddt\,z(s , t , -1)\right)^2+\frac{1}{4}\dy^2\,\left(\dds\,z(s , t , -1)\right)^2+\frac{1}{16}\dx^2\,\dy^2}, \\
  \left| \N_{6}\right|  & = \sqrt{\frac{1}{4}\dx^2\,\left(\ddt\,z(s , t , 1)\right)^2 +\frac{1}{4}\dy^2\,\left(\dds\,z(s , t , 1)\right)^2 +\frac{1}{16}\dx^2\,\dy^2}. \\
\end{align*}

\bibliography{references}
\bibliographystyle{plain}

\end{document}