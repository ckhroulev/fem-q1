% Created 2015-09-16 Wed 21:53
\documentclass[11pt]{article}
\usepackage[utf8]{inputenc}
\usepackage[T1]{fontenc}
\usepackage{fixltx2e}
\usepackage{graphicx}
\usepackage{longtable}
\usepackage{float}
\usepackage{wrapfig}
\usepackage{rotating}
\usepackage[normalem]{ulem}
\usepackage{amsmath}
\usepackage{textcomp}
\usepackage{marvosym}
\usepackage{wasysym}
\usepackage{amssymb}
\usepackage{hyperref}
\tolerance=1000
\usepackage[margin=1in]{geometry}
\parindent=0in \parskip=0.5\baselineskip
\author{Constantine Khroulev}
\date{\today}
\title{Evaluating line and surface integrals to implement boundary conditions}

\begin{document}

\maketitle
\tableofcontents

\newcommand{\dx}{\Delta x}
\newcommand{\dy}{\Delta y}
\newcommand{\Real}{\mathbb{R}}
\newcommand{\diff}[2]{\frac{\partial #1}{\partial #2}}
\newcommand{\rs}{\R_{s}}
\newcommand{\rt}{\R_{t}}
\newcommand{\R}{\vec r}
\newcommand{\N}{\vec n}
\newcommand{\rRef}{\R_{*}}

\section{Introduction}
\label{sec-1}

To implement Neumann boundary conditions in a FEM solver we need to be
able to evaluate integrals over element boundaries. In the
two-dimensional case this requires computing \emph{line integrals}; in
the three-dimensional case --- \emph{surface integrals}. In these
notes I describe how to use element maps to approximate these
integrals using quadratures.

Karniadakis and Sherwin mention this in section 4.1.4 (page 160) of
\cite{KarniadakisSherwin}, but use an unhelpful name \emph{surface
  Jacobian} and don't go into details.

\section{Preliminaries}
\label{sec-2}

\subsection{Line integrals}
\label{sec-2-1}

For some scalar function $F : U \subset \Real^{n} \to \Real$
the integral along a piecewise smooth curve $C \subset U$ is defined as
\begin{equation}
\label{eq:1}
\int_{C} F\, ds = \int_{a}^{b} F(\R(t))\, \left| \frac{d}{dt}\R(t) \right|\, dt,
\end{equation}
where $\R: [a, b] \to C$ is a bijective parameterization of the curve
$C$ \cite{Nikolsky1977}.

\subsection{Surface integrals}
\label{sec-2-2}

Let $\Omega \subset \Real^{2}$ be a measurable subset of the
plane and define $\R(s,t) = (x(s,t), y(s,t), z(s,t))$, where $x(s,
t)$, $y(s, t)$, and $z(s, t)$ are continuously differentiable
functions defined on $\Omega$. Then $S = \R(\Omega) \subset
\Real^{3}$ is a smooth surface parameterized by $\R$.

Vectors $\rs$ and $\rt$ are tangential vectors to the surface $S$
given by the parameterization $\R(s,t)$.

If $\rs$ and $\rt$ are not parallel, then $\N = \rs \times \rt$ is
a normal vector, $| \N | > 0$, and we can define
the integral
\begin{equation}
\label{eq:2}
\int_{S} F\, dS = \int_{\Omega} F\, |\N|\, ds dt.
\end{equation}

Here
\begin{equation}
\label{eq:3}
\rs = \left(\diff{x(s,t)}{s}, \diff{y(s,t)}{s}, \diff{z(s,t)}{s}\right) \quad \text{and} \quad
\rt = \left(\diff{x(s,t)}{t}, \diff{y(s,t)}{t}, \diff{z(s,t)}{t}\right).
\end{equation}
See \cite{Nikolsky1977} or any other Calculus textbook for details.


\section{Boundaries of quadrilateral elements}
\label{sec-3}

A 2D FEM construction includes the map from the reference element
$E_{*} = [-1, 1] \times [-1,1]$ to a physical element:

\begin{equation}
\label{eq:4}
\rRef(\xi,\eta) = \left( x(\xi,\eta), y(\xi,\eta) \right).
\end{equation}

A side of the reference element is mapped to the corresponding side of
the physical element. We number faces as follows.
\begin{equation}
\label{eq:5}
\begin{aligned}
C_{*,1} &=  E_{*} \cap \left \{ (\xi, \eta)\, |\, \xi = -1 \right \},\\
C_{*,2} &=  E_{*} \cap \left \{ (\xi, \eta)\, |\, \xi = 1  \right \},\\
C_{*,3} &=  E_{*} \cap \left \{ (\xi, \eta)\, |\, \eta = -1 \right \},\\
C_{*,4} &=  E_{*} \cap \left \{ (\xi, \eta)\, |\, \eta = 1  \right \}.
\end{aligned}
\end{equation}
Now sides of a physical element are $C_{E,i} = \rRef(C_{*,i})$. The
face $C_{E,1}$ is parameterized by
\begin{equation}
\label{eq:6}
\R_{E,1}(t) = \rRef(-1, t),
\end{equation}
and so on.

\subsection{Example: evaluating integrals over sides of $Q_{1}$ quad elements}
\label{sec-3-1}

In this example I use the first ("left") side, but computations for
all sides are similar.

Let $(\xi_{i}, \eta_{i}), i = 1,\dots,4$ be coordinates of the nodes
of the reference element $E_{*}$ and $(x_{i}, y_{i})$ be coordinates
of the nodes of a physical element $E$.

Let $F(x,y)$ be a function defined on the FEM mesh and on $C_{E,1}$ in
particular. Assume that we know its nodal values
\begin{equation}
\label{eq:7}
F_{i} = F(x_{i}, y_{i}).
\end{equation}

Our goal is to approximate the integral
\begin{equation}
\label{eq:8}
I = \int_{C_{E,1}} F(x,y)\, ds
\end{equation}

In this case the map $\rRef$ is defined by \eqref{eq:4} and
\begin{equation}
\label{eq:9}
\begin{aligned}
x(\xi,\eta) &= \sum_{i=1}^{4} x_{i}\, \chi_{i}(\xi, \eta),\\
y(\xi,\eta) &= \sum_{i=1}^{4} y_{i}\, \chi_{i}(\xi, \eta),\\
\end{aligned}
\end{equation}
where
\begin{equation}
\label{eq:10}
\chi_{i}(\xi,\eta) = \frac14 (1 + \xi\xi_{i})(1 + \eta\eta_{i})
\end{equation}
are $Q_{1}$ reference element basis functions.

Now we can rewrite $I$
\begin{equation}
\label{eq:11}
I = \int_{-1}^{1} F(x(-1, t), y(-1, t))\, \left|\frac{d}{dt}\R_{1}(t) \right|\, dt.
\end{equation}

Here
\begin{align}
\label{eq:12}
\R_{1}(t) &= \left( x(-1,t),\, y(-1,t) \right),\\
\frac{d}{dt} \R_{1}(t) &= \left( \sum_{i=1}^{4} x_{i}\, \frac{d}{dt}\chi_{i}(-1, t),\,
                                 \sum_{i=1}^{4} y_{i}\, \frac{d}{dt}\chi_{i}(-1, t)  \right).
\end{align}

The integral in \eqref{eq:11} is a \emph{regular} definite integral over
the interval $[-1, 1]$ and can be approximated using a quadrature.

\begin{equation}
\label{eq:13}
I \approx \sum_{q=1}^{N_{q}} w_{q}\, F(x(-1, t_{q}), y(-1, t_{q}))\, \left|\frac{d}{dt}\R(t_{q}) \right|.
\end{equation}

Now we just need to approximate $F(x(-1, t_{q}), y(-1, t_{q}))$ by
a bilinear function built on \eqref{eq:10}:
\begin{equation}
\label{eq:14}
F(x(-1, t), y(-1, t)) \approx \sum_{j=1}^{4} F_{j}\, \chi_{j}(-1, t)
\end{equation}
so that
\begin{equation}
\label{eq:15}
I \approx \sum_{q=1}^{N_{q}} w_{q}\,
\left( \sum_{j=1}^{4} F_{j}\, \chi_{j}(-1, t_{q}) \right) \,
\left| \frac{d}{dt}\R(t_{q}) \right|.
\end{equation}

In the equal spacing case ($x_{4} = x_{1}$, $x_{2} = x_{3} = x_{1} +
\dx$, $y_{2} = y_{1}$, $y_{3} = y_{4} = y_{1} + \dy$ everything is much simpler:
\begin{align*}
\left| \frac{d}{dt}\R_{1}\right|  & = \frac{\dy}{2}, \\
\left| \frac{d}{dt}\R_{2}\right|  & = \frac{\dy}{2}, \\
\left| \frac{d}{dt}\R_{3}\right|  & = \frac{\dx}{2}, \\
\left| \frac{d}{dt}\R_{4}\right|  & = \frac{\dx}{2}. \\
\end{align*}

\section{Boundaries of hexahedral elements}
\label{sec-4}

A 3D FEM construction includes the map from the reference element
$E_{*} = [-1, 1] \times [-1,1] \times [-1,1]$ to a physical element:

\begin{equation}
\label{eq:16}
\rRef(\xi,\eta,\zeta) = \left( x(\xi,\eta,\zeta), y(\xi,\eta,\zeta), z((\xi,\eta,\zeta) \right)
\end{equation}

Note that $\rRef$ maps a face of the reference element to the
corresponding face of a physical element. We number the faces of the
reference element as follows.

\begin{equation}
\label{eq:17}
\begin{aligned}
S_{*,1} &=  E_{*} \cap \left \{ (\xi, \eta, \zeta)\, |\, \xi = -1 \right \},\\
S_{*,2} &=  E_{*} \cap \left \{ (\xi, \eta, \zeta)\, |\, \xi = 1  \right \},\\
S_{*,3} &=  E_{*} \cap \left \{ (\xi, \eta, \zeta)\, |\, \eta = -1 \right \},\\
S_{*,4} &=  E_{*} \cap \left \{ (\xi, \eta, \zeta)\, |\, \eta = 1  \right \},\\
S_{*,5} &=  E_{*} \cap \left \{ (\xi, \eta, \zeta)\, |\, \zeta = -1 \right \},\\
S_{*,6} &=  E_{*} \cap \left \{ (\xi, \eta, \zeta)\, |\, \zeta = 1  \right \}.
\end{aligned}
\end{equation}

Then faces of a physical element are $S_{E,i} = \rRef(S_{*,i})$.
The face $S_{E,1}$ is parameterized by
\begin{equation*}
\label{eq:18}
\R_{E,1}(s, t) = \rRef(-1,s,t),
\end{equation*}
and so on.

\subsection{Example: evaluating an integral over a face of a $Q_{1}$ element.}
\label{sec-4-1}

\newcommand{\face}{S_{E,5}}

In this example I use the fifth, or "bottom" face, but all
computations are very similar in the five other cases.

Let $(\xi_{i}, \eta_{i}, \zeta_{i})$, $i = 1,\dots,8$ be coordinates
of the nodes of the reference element $E_{*}$ and $(x_{i}, y_{i},
z_{i})$ the coordinates of the nodes of a physical element $E$.

Let $F(x,y,z)$ be a function defined on the FEM mesh and on $\face$
in particular. Assume that we know its nodal values
\begin{equation*}
F_{i} = F(x_{i}, y_{i}, z_{i}).
\end{equation*}

Our goal is to approximate the integral
\begin{equation}
\label{eq:19}
I = \int_{\face} F(x,y,z)\, dS.
\end{equation}

In this case the map $\rRef$ is defined by \eqref{eq:16} and
\begin{equation}
\label{eq:20}
\begin{aligned}
x(\xi,\eta,\zeta) &= \sum_{i=1}^{8} x_{i}\, \chi_{i}(\xi, \eta, \zeta),\\
y(\xi,\eta,\zeta) &= \sum_{i=1}^{8} y_{i}\, \chi_{i}(\xi, \eta, \zeta),\\
z(\xi,\eta,\zeta) &= \sum_{i=1}^{8} z_{i}\, \chi_{i}(\xi, \eta, \zeta),
\end{aligned}
\end{equation}
where
\begin{equation}
\label{eq:21}
\chi_{i}(\xi,\eta,\zeta) = \frac18 (1 + \xi\xi_{i})(1 + \eta\eta_{i})(1 + \zeta\zeta_{i})
\end{equation}
are $Q_{1}$ reference element basis functions.

We can now rewrite $I$ using \eqref{eq:16}, \eqref{eq:20}, and the definition of the surface integral:
\begin{equation}
\label{eq:22}
I = \int_{E_{*}} F(x(s,t,-1), y(s,t,-1), z(s,t,-1))\, | \N(s,t) |\,dsdt,
\end{equation}
where $\N(s,t) = \rs(s,t) \times \rt(s,t)$ and
\begin{equation}
\label{eq:23}
\begin{aligned}
\rs(s,t) &= \left(\diff{x(s,t,-1)}{s}, \diff{y(s,t,-1)}{s}, \diff{z(s,t,-1)}{s}\right),\\
\rt(s,t) &= \left(\diff{x(s,t,-1)}{t}, \diff{y(s,t,-1)}{t}, \diff{z(s,t,-1)}{t}\right),
\end{aligned}
\end{equation}
since the face $\face$ is parameterized by $\R_{E,5}(s, t) = \rRef(s, t, -1)$.

The integral $I$ written in the form \eqref{eq:22} can be approximated
using a quadrature on the reference element, i.e.
\begin{equation}
\label{eq:24}
I \approx \sum_{q=1}^{N_{q}} w_{q}\cdot F(x(s_{q},t_{q},-1), y(s_{q},t_{q},-1), z(s_{q},t_{q},-1)) | \N(s_{q},t_{q}) |,
\end{equation}
where $w_{q}$ are quadrature weights and $s_{q}, t_{q}$ are
coordinates of quadrature points.

To compute this quadrature we need
$F(x(s_{q},t_{q},-1), y(s_{q},t_{q},-1), z(s_{q},t_{q},-1))$, but we
assume that only nodal values of $F$ are available.

We use \eqref{eq:21} as a basis to write a trilinear approximation of $F$ on $\face$:
\begin{equation}
\label{eq:25}
F(s,t) \approx \sum_{k=1}^{8} F_{k}\,\chi_{k}(s, t, -1),
\end{equation}
then
\begin{equation}
\label{eq:26}
I \approx \sum_{q=1}^{N_{q}} w_{q}\left( \sum_{k=1}^{8} F_{k}\,\chi_{k}(s_{q}, t_{q}, -1) \right) | \N(s_{q},t_{q}) |,
\end{equation}

Finally, to evaluate $|\N(s_{q}, t_{q})| = | \rs(s_{q}, t_{q}) \times
\rt(s_{q}, t_{q}) |$ we use \eqref{eq:3} and \eqref{eq:20}.
The vector $\rs(s_{q}, t_{q})$ is defined by
\begin{equation}
\label{eq:27}
\rs(s_{q},t_{q}) = \left(
  \sum_{j=1}^{8} x_{j} \left.\diff{\chi_{j}(s, t, -1)}{s}\right|_{(s_{q}, t_{q})},\,
  \sum_{j=1}^{8} y_{j} \left.\diff{\chi_{j}(s, t, -1)}{s}\right|_{(s_{q}, t_{q})},\,
  \sum_{j=1}^{8} z_{j} \left.\diff{\chi_{j}(s, t, -1)}{s}\right|_{(s_{q}, t_{q})} \right),
\end{equation}
similarly for $\rt(s_{q}, t_{q})$.

Overall, here are the steps needed to approximate \eqref{eq:19}.

\begin{enumerate}
\item Pre-compute $\chi_{j}(s_{q}, t_{q}, -1)$ for $\face$ and similarly
for the remaining five faces.
\item Pre-compute $\diff{\chi_{i}(s, t, -1)}{s}$ and
$\diff{\chi_{i}(s, t, -1)}{t}$ and $(s_{q}, t_{q})$ for $\face$ and similarly for
other faces.
\end{enumerate}

Then for each quadrature point we need to

\begin{enumerate}
\item use pre-computed values to approximate $\rs(s_{q},t_{q})$ and
$\rt(s_{q},t_{q})$ (see \eqref{eq:27}),
\item compute $|\N(s_{q}, t_{q})| = | \rs(s_{q}, t_{q}) \times \rt(s_{q}, t_{q}) |$,
\item approximate $F(s_{q}, t_{q})$ using \eqref{eq:25},
\item multiply to evaluate a term of \eqref{eq:26}.
\end{enumerate}

\subsubsection{Equal horizontal spacing}
\label{sec-4-1-1}

The procedure described above can be simplified if we know that the
FEM mesh uses regular spacing in the $x$ and $y$ directions, i.e.
$x$ and $y$ coordinates of the nodes of a physical element are
\begin{equation}
  \label{eq:28}
  \begin{array}{rcllllllll}
    x_{i} &=& \{ x_{1}, &x_{1} + \dx, &x_{1} + \dx, &x_{1}, &x_{1}, &x_{1} + \dx, &x_{1} + \dx, &x_{1} \},\\
    y_{i} &=& \{ y_{1}, &y_{1}, &y_{1} + \dy, &y_{1} + \dy, &y_{1}, &y_{1}, &y_{1} + \dy, &y_{1} + \dy \};\\
  \end{array}
\end{equation}
$z_{i}$ are may all be different.

In this case \eqref{eq:27} simplifies to
\begin{equation}
  \label{eq:29}
  \rs(s,t) = \left(
    \frac{\dx}{2},\,
    0,\,
    \sum_{i=1}^{8} z_{i} \diff{\chi_{i}(s, t, -1)}{s} \right),
\end{equation}
and $\rt(s,t)$ is given by
\begin{equation}
  \label{eq:30}
  \rt(s,t) = \left(
    0,\,
    \frac{\dy}{2},\,
    \sum_{i=1}^{8} z_{i} \diff{\chi_{i}(s, t, -1)}{t} \right).
\end{equation}

With these simplifications $\N(s,t)$ becomes

\begin{equation}
  \label{eq:31}
  \N(s,t) = \left(
    -\frac{\dy}{2}\diff{z(s,t,-1)}{s},
    -\frac{\dx}{2}\diff{z(s,t,-1)}{t},
    \frac{1}{4} \dx \dy
  \right)
\end{equation}

Note that if all $z_{i}$ are equal, $|\N| = \frac{1}{4} \dx \dy$ is
equal to the determinant of the Jacobian of the map from the reference
element to a physical element in the 2D $Q_{1}$ FEM setup with equal
spacing in $x$ and $y$ directions --- as expected.

The only thing that changes depending on the face we're integrating
over is the expression for $|\N|$. Below are formulas for all six
faces (assuming equal horizontal spacing and using the numbering
defined by \eqref{eq:17}).

\newcommand{\ddt}{\frac{d}{d\,t}}
\newcommand{\dds}{\frac{d}{d\,s}}
\begin{align*}
  \begin{array}{rcccl}
    \N_{1} = \Big( & \frac{1}{2}\dy\ddt\,z(-1 , s , t),& 0 ,& 0 & \Big) , \\
    \N_{2} = \Big( & \frac{1}{2}\dy\ddt\,z(1 , s , t),& 0 ,& 0 & \Big) , \\
    \N_{3} = \Big( & 0 ,& -\frac{1}{2}\dx\,\ddt\,z(s , -1 , t),& 0 & \Big) , \\
    \N_{4} = \Big( & 0 ,& -\frac{1}{2}\dx\,\ddt\,z(s , 1 , t),& 0 & \Big) , \\
    \N_{5} = \Big( & -\frac{1}{2}\dy\,\dds\,z(s , t , -1) ,& -\frac{1}{2}\dx\,\ddt\,z(s , t , -1) ,& \frac{1}{4}\dx\,\dy & \Big) , \\
    \N_{6} = \Big( & -\frac{1}{2}\dy\,\dds\,z(s , t , 1) ,& -\frac{1}{2}\dx\,\ddt\,z(s , t , 1) ,& \frac{1}{4}\dx\,\dy & \Big) . \\
  \end{array}
\end{align*}

\begin{align*}
  \left| \N_{1}\right|  & = \frac{1}{2}\dy\left| \ddt\,z(-1 , s , t)\right|, \\
  \left| \N_{2}\right|  & = \frac{1}{2}\dy\left| \ddt\,z(1 , s , t)\right|, \\
  \left| \N_{3}\right|  & = \frac{1}{2}\dx\left| \ddt\,z(s , -1 , t)\right|,\\
  \left| \N_{4}\right|  & = \frac{1}{2}\dx\left| \ddt\,z(s , 1 , t)\right|, \\
  \left| \N_{5}\right|  & = \sqrt{\frac{1}{4}\dx^2\,\left(\ddt\,z(s , t , -1)\right)^2+\frac{1}{4}\dy^2\,\left(\dds\,z(s , t , -1)\right)^2+\frac{1}{16}\dx^2\,\dy^2}, \\
  \left| \N_{6}\right|  & = \sqrt{\frac{1}{4}\dx^2\,\left(\ddt\,z(s , t , 1)\right)^2 +\frac{1}{4}\dy^2\,\left(\dds\,z(s , t , 1)\right)^2 +\frac{1}{16}\dx^2\,\dy^2}. \\
\end{align*}

\bibliography{references}
\bibliographystyle{plain}

\end{document}